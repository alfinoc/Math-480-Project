\documentclass{article}
\usepackage{multicol}
\usepackage{titlesec}
\usepackage[margin=0.75in]{geometry}

% MACROS %

\titleformat{\section}
  {\large\scshape\filcenter}{\thesection}{}{}

\begin{document}

\title{\textbf{Optimal Scheduling of Introductory Programming Lab}}
\author{
   Chris Alfino \textit{\&} Rajan Selvan\\
   \vspace{12pt}
   \small\textit{Math 480}\\
   Connor Moore\\
   \small\textit{CSE TA Coordinator}}
\date{}

\maketitle
\noindent\makebox[\linewidth]{\rule{\textwidth}{0.4pt}}

\setlength\columnsep{0.45in}
\setlength{\parskip}{0.5em}
\begin{multicols}{2}

\section*{Background}

In recent years, enrollment in the University of Washington's introductory computer science classes, CSE 142 and CSE 143, has skyrocketed. Students are interested in programming for a variety of reasons; industry and academic research positions abound, while programming skills are becoming increasingly necessary in other disciplines around campus. In Winter 2014, CSE 142 and 143 each hit new enrollment records: 810 and 530 students respectively! In order to meet new demand, CSE 142 and 143 have needed to scale their educational efforts. Requisite hiring of undergraduate teaching assistants has increased in proportion with enrollment, with each TA-lead discussion section containing roughly 20 students.

The introductory programming lab in Mary Gates Hall is a two room computer lab open from 12:30pm to 9:30pm every week day and from 1:30pm to 3:30pm on Saturday and Sunday. Students in CSE 142 and 143 can come to the lab during these hours to be helped with conceptual and homework questions by whichever TAs are currently staffed. A queuing system in the lab allows a student to add him/herself to a help queue through a web interface. The process for a working TA is then straightforward: he/she dequeues a student and helps the student one-on-one for 2 to 10 minutes. A TA typically signs up for 2 one-hour schedule slots each week, so that the number of TAs in the lab ranges from as few as 1 or 2 on weekends to as many as 7 on busy weekday evenings.

\section*{Objective}

At a high level, the objective of our model is to schedule TAs such that average student wait time in the queue is minimized subject to constraints like the following:
\begin{itemize}
    \item The IPL is always staffed with at least one TA.
    \item There is at least one senior TA staffed during key hours before due dates for the more advanced assignments. A \textit{senior TA} is a TA that has instructed for at least three quarters.
    \item TA schedule preferences are accomodated wherever possible.
    \item Minimum and maximum 
\end{itemize}

The CS department has already done some work to accommodate varying student demand across the week. For instance, more students arrive at the IPL during the hours before important due dates, so more TAs are staffed during these hours. Over the past couple years, data has been collected about the frequency of CSE 142 and 143 student questions at various times of the day.

TAs themselves have preferences that cannot be ignored in scheduling hours. Each student has some hours that they cannot work (during class, for instance), and also preferences about when they'd like to work. As a goal, we would like to maximize TA preference accommodation, splitting ties according to TA seniority (number of quarters taught). These preferences must be balanced, however, with the need to compose groups of ideally mixed seniority during key hours. Before a particularly tough CSE 143 assignment, it is essential that more experienced TAs are present to field questions and help less experienced TAs. These factors should we weighted into the objective function in a way that is customizable in the future.

\section*{Approach}
In some ways, this is a standard linear programming problem, similar to scheduling problems we've already seen in class. For the LP formulation below, data will come in the form of:
\begin{itemize}
    \item TA preference/schedule requirements, where each TA provides a real number from 0 to 1 (unavailable to available) for each time slot.
    \item TA experience levels, including number of quarters TA'd for each of CSE 142 and 143.
    \item CSE Department requirements for minimum and maximum number of total and senior TAs needed for each slot.
\end{itemize}

The precise formulation of the LP model is described below. It makes use of several known constants: the minimum total and senior TAs required per time slot. The CSE department has already estimated these lower bounds from the number of student questions per hour, and their estimates should work fine for initial solutions of our LP. But in order to generate better estimates, we plan to build a simulation. Similar to how the American West researchers discovered additional features of airplane boarding queueing behavior by constructing a simulation, we hope to gain insights into the causes of IPL queue backlog. It is possible, for instance, that staffing fewer, experienced TAs moves the student queue faster than many, less experienced TAs, since average help time could be minimized.

Technically, students are supposed to be helped for at most 2 or 10 minutes, depending on which queue they add themselves to in the IPL. But the assumption that help times obey these guidelines is likely unrealistic. Early testing should allow us to collect one more category of data: true amount of time required to serve each student question. These times should be easy to collect by observing TA behavior during sample hours during the week. We expect that queue backlog will be more significant during peak hours, so we may restrict data collection to a small window (the final hours of Thursday night immediately before a due date, perhaps). 

\section*{Basic LP}
Our problem can be formulated as a general asssignment problem, with $m$ TAs as supply points and $n$ time slots as demand points. Under this formulation, we let $x_{ij}$ be a 0-1 variable where $x_{ij} = 1$ if TA $i$ works time slot $j$ and $x_{ij} = 0$ otherwise.

For each TA $i$, let $s_i = 1$ if TA $i$ is a senior TA, 0 otherwise. Each TA must work at least two hours each week and can work up to a self-imposed maximum $m_i$ hours. As undergraduate UW employees, TAs cannot legally work more than 19.5 hours. Formally for each TA $i$,

\begin{equation}
2 \leq \sum_{j=1}^{n}x_{ij} \leq m_i \leq 19
\end{equation}

Time slot preferences contain an addition contraint. In particular, if real number $a_{ij} \in [0,1]$ represents the availability preference of TA $i$ at slot $j$, then if $a_{ij} = 0$, we intrepret this preference as absolute: the TA \textit{cannot} work the time slot, so should not be scheduled for that slot. Using the usual techniques for if-then constraints, we require for each TA $i$ and time slot $j$,
\begin{equation}
x_{ij} = 1 \textrm{ iff } 0 < a_{ij}
\end{equation}

For each time slot $j$, let $t_j$ and $u_j$ be the minimum number of total and senior TAs respectively required. Let $T_j$ be the maximum number of TAs allowed for each time slot. Formally for each time slot $j$,

\begin{equation}
t_j \leq \sum_{i=1}^{m}x_{ij} \leq T_j
\end{equation}

\begin{equation}
u_j \leq \sum_{i=1}^{m}x_{ij}s_i
\end{equation}

Unlike a general transportation problem, we are not simply minimizing supply usage. After all, TAs may in fact \textit{wish} to work more hours, and since the CSE department has money to spare on staffing costs, we can assume that the department is willing to pay to staff as many as $T_j$ TAs in each slot. Our objective function is a measure of average TA preference, weighted by TA seniority. The adverage adherence to the preferences of TA $i$ is,

\begin{equation}
a(i) = \frac{1}{n}\sum_{j=1}^na_{ij}x_{ij}
\end{equation}

For each TA $i$, let $q_i$ be the number of quarters that TA has taught (a measure of seniority) and write $q_{max} = max\{q_1, ..., q_m\}$. Our objective function is the weighted average,

\begin{equation}
\textrm{maximize } \frac{1}{m\cdot q_{max}} \sum_{i=1}^mq_i\cdot a(i)
\end{equation}

\section*{Resources}
As we translate our IPL staffing problem into a linear programming problem, there are various utilities available to begin constructing our model. Between SageMath and the Wolfram Language, we have access to functions for maximizing and minimizing our objective function subject to our constraints. As an assignment problem, our objective and the Hungarian solution is treated in section 7.5 of Winston (page 393).

The CS department makes the current and past IPL schedules public online. This information will be useful in determining our baseline staffing values, as well as initially providing us with a sense of average question frequency versus number of TAs available during that a period. In addition, assignment due date schedules for CSE 142 and CSE 143 will of course be helpful in determining when higher volumes of students are ex-pected in the IPL. And finally, the CS department stores constantly updating data on the number of questions asked per hour and minutes of waittime per question.

\section*{Conclusion}
We hope to model the problem of optimal TA staffing first as a linear programming problem similar to other scheduling problems we've seen. From there, we will build a simulation to model the more probabilistic nature of the lab’s student queue. The problem solution, a set of staffing assignments, should find immediate application, and hopefully, we'll have an opportunity to test our resulting approach in future quarters.

\end{multicols}
\end{document}
